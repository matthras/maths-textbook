Geometric sequences are sequences where each term differs by the same ratio (referred to as the common ratio). Examples of geometric sequences include:

\begin{itemize}
  \item $1, \frac{1}{2}, \frac{1}{4}, \frac{1}{8}, \dots$
  \item $3, 9, 27, 81, \dots$ (powers of 3)
\end{itemize}

A geometric sequence has two key components that allow us to determine all of its terms: 

\begin{itemize}
  \item a starting term, denoted $t_1$ ($a$ in other resources), and
  \item a common ratio, denoted $r$.
\end{itemize}

\BoxedExample{
  Using the same examples as listed at the beginning of the chapter:
  \begin{center}
    \begin{tabular}{|l|c|c|}
      \hline
      Geometric Sequence & $t_1$ & $r$ \\
      \hline
      $1, \frac{1}{2}, \frac{1}{4}, \frac{1}{8}, \dots$ & 1 & $\frac{1}{2}$ \\
      $3, 9, 27, 81, \dots$ & 3 & 3 \\
      \hline
    \end{tabular}
  \end{center}
}