Arithmetic sequences are sequences where each term differs by the same amount (referred to as the common difference). Examples of arithmetic sequences include:

\begin{itemize}
  \item $1, 3, 5, 7, 9, 11, \dots$ (odd numbers)
  \item $2, 4, 6, 8, 10, 12, \dots$ (even numbers)
  \item $5, 10, 15, 20, 25, 30, 35, \dots$ (multiples of 5)
  \item $-4, -1, 2, 5, 8, 11, \dots$ 
\end{itemize}

An arithmetic sequence has two key components that allow us to determine all of its terms:

\begin{itemize}
  \item a starting term, denoted $t_1$ ($a$ in other resources), and
  \item a common difference, denoted $d$.
\end{itemize}

\boxedExample{
  Using the same examples as listed at the beginning of the chapter:
  \begin{center}
    \begin{tabular}{|l|c|c|}
      \hline
      Arithmetic Sequence & $t_1$ & $d$ \\
      \hline
      $1, 3, 5, 7, 9, 11, \dots$ & 1 & 2 \\ 
      $2, 4, 6, 8, 10, 12, \dots$ & 2 & 2 \\
      $5, 10, 15, 20, 25, 30, 35, \dots$ & 5 & -4 \\
      $-4, -1, 2, 5, 8, 11, \dots$ & -4 & 3 \\
      \hline
    \end{tabular}
  \end{center}
}